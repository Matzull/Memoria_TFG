% Fichero con acr�nimos para su uso con GlossTeX
%
% El listado de acr�nimos funciona de manera similar a las referencias
% bibliogr�ficas. Este  fichero es equivalente por tanto  al .bib, con
% el listado de  los acr�nimos disponibles. A lo  largo del documento,
% cuando se  utilice alguno  de ellos, se  referencia para  indicar al
% sistema que queremos que dicho  acr�nimo aparezca en el glosario. En
% contra de  lo que ocurre con  la bibliograf�a, el  texto original no
% sufre ninguna modificaci�n.
%
% Las entradas son:
% @entry{<etiqueta>[,<item>[,<forma larga>]]} [<texto>]
%
% La etiqueta es el  nombre del acr�nimo (equivalente a la etiqueta en
% BibTeX) que luego se utiliza para "referenciarlo" en el texto.

% Para  que  la  generaci�n   en  Release  con  el  Makefile  funcione
% correctamente, necesitar�s  al menos tener  referenciado un acr�nimo
% dentro del  texto.  Si  no quieres usar  acr�nimos, o de  momento no
% tienes ninguno  referenciado, basta con que no  definas la constante
% \acronimosEnRelease en config.tex


\newacronym{aprs}{APRS}{Automatic Packet Reporting System}

\newacronym{sbc}{SBC}{Single Board Computer}

\newacronym{aprs-is}{APRS-IS}{Automatic Packet Reporting System Internet Service}

\newacronym{rtl-sdr}{RTL-SDR}{Realtek Software Defined Radio}

\newacronym{aws}{AWS}{Amazon Web Services}

\newacronym{json}{JSON}{JavaScript Object Notation}

\newacronym{ssh}{SSH}{Secure SHell}

\newacronym{http}{HTTP}{HyperText Transfer Protocol}

\newacronym{wsgi}{WSGI}{Web Server Gateway Interface}

\newacronym{osint}{OSINT}{Open Source Intelligence}

\newacronym{tnc}{TNC}{Terminal Node Controller}

\newacronym{gps}{GPS}{Global Positioning System}

\newacronym{api}{API}{Application Programming Interface}

\newacronym{ORM}{ORM}{Object-Relational Mapping}

\newacronym{SQL}{SQL}{Structured Query Language}

\newacronym{DDNS}{DDNS}{Dynamic Domain Name System}

\printunsrtglossary[type=\acronymtype, title=Acrónimos, toctitle=Lista de acrónimos]

